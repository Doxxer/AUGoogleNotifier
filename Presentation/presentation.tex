\documentclass[10pt,pdf,hyperref={unicode}]{beamer}
\usepackage[utf8]{inputenc}
\usepackage[russian]{babel}
\usepackage{graphicx}
\usetheme{Darmstadt}

\usenavigationsymbolstemplate{}

\begin{document}

    \title[GoogleNotifier]{Гугл-уведомлятор}
    \subtitle{НИР, СПбАУ, весна 2014}
    \author{Тимур Тураев, Богдан Бугаев \\ Руководитель: Евгений Краско}
    \date{26 мая 2014 г.}

    \frame{\titlepage}

    \begin{frame}\frametitle{План}
        \begin{itemize}
        	\item О проекте
            \item Задачи
            \item Проблемы
            \item Результаты
            \item Полученные знания
        \end{itemize}
    \end{frame}
    
    \begin{frame}\frametitle{О проекте}
        \begin{itemize}
            \item	PlaceHolder (TT)
        \end{itemize}
    \end{frame}    

    \begin{frame}\frametitle{Задачи}
        \begin{itemize}%[<+->]
            \item	TO KRASKO: какие задачи описывать -- те, что стояли изначально (то есть, написать сервер, написать клиенты)
            \item   Разобраться с серверной часть PL
            \item   Клиенты
            \item   (*) Мобильная платформа
        \end{itemize}
    \end{frame}
    
    \begin{frame}\frametitle{Проблемы}
        \begin{itemize}%[<+->]
            \item   PL (серверная часть) (TT)
        \end{itemize}
    \end{frame}
    
    \begin{frame}\frametitle{Результаты}
        \begin{itemize}%[<+->]
            \item	PL (серверная часть) (TT)
        \end{itemize}
    \end{frame}
    
        \begin{frame}\frametitle{Проблемы (клиентская часть)}
        \begin{itemize}%[<+->]
            \item   Особенности Qt 5 под Ubuntu (Unity)
            \item   Особенности кросс-платформенной разработки в Qt (Linux/Mac OS X/Windows)
        \end{itemize}
    \end{frame}
    
    \begin{frame}\frametitle{Результаты (клиентская часть)}
        \begin{itemize}%[<+->]
            \item	Разработано кросс-платформенное ядро клиента            	
            \item	Консольный прототип клиента для тестирования функциональности
            \item	Разработаны platform-spisific клиенты с графическим интерфейсом
        \end{itemize}
    \end{frame}

    \begin{frame}\frametitle{Полученные знания}
        \begin{itemize}%[<+->]
            \item	PL (серверная часть)
            \item	PL (клиентская часть)
        \end{itemize}
    \end{frame}

    \begin{frame}
        \begin{center}
            Спасибо за внимание!
        \end{center}
    \end{frame}

\end{document}
