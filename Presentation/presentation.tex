\documentclass[10pt,pdf,hyperref={unicode}]{beamer}
\usepackage[utf8]{inputenc}
\usepackage[russian]{babel}
\usepackage{graphicx}
\usetheme{Darmstadt}

\usenavigationsymbolstemplate{}

\begin{document}

    \title[GoogleNotifier]{SPbAU Google Notifier}
    \subtitle{НИР, СПбАУ, весна 2014}
    \author{Тимур Тураев, Богдан Бугаев \\ Руководитель: Евгений Краско}
    \date{26 мая 2014 г.}

    \frame{\titlepage}

    \begin{frame}\frametitle{План}
        \begin{itemize}
        	\item О проекте
            \item Задачи
            \item Проблемы
            \item Результаты
            \item Полученные знания
        \end{itemize}
    \end{frame}
    
    \begin{frame}\frametitle{О проекте}
        \begin{itemize}
            \item	Google позволяет отследить изменения одного конкретного документа
            \pause
            \item	На изменения нужно подписываться вручную
            \pause            
            \item	Это не очень удобно, когда документов много
            \pause            
            \item	Хочется автоматически следить за всеми документами разом
            \pause            
            \item	Кросплатформенность
        \end{itemize}
    \end{frame}    

    \begin{frame}\frametitle{Задачи}
        \begin{itemize}%[<+->]
            \item	TO KRASKO: какие задачи описывать -- те, что стояли изначально (то есть, написать сервер, написать клиенты) или конкретно: написать модуль авторизации, написать модуль подписки, модуль граббера
            \item   Разобраться с серверной часть PL (?)
            \item   Клиенты
            \item   (*) Мобильная платформа
        \end{itemize}
    \end{frame}
    
    \begin{frame}\frametitle{Проблемы}
        \begin{itemize}%[<+->]
	        \item   Google App Engine
	        \pause	        
            \item   Особенность Google Drive API
            \begin{itemize}
                \item иформация в NotificationCallback
                \item обновление подписки
                \item регистрация сайта
                \item billing                
            \end{itemize}
        \end{itemize}
    \end{frame}
    
    \begin{frame}\frametitle{Результаты}
        \begin{itemize}%[<+->]
            \item	Написан веб-сервер, обрабатывающий запросы, поступающие от клиентов            
            \item	Авторизация, аутентификация, сессии, защита cookie
	        \pause
            \item	Модуль подписок (с автообновлением)
            \item	Сохранение информации об изменениях в базу данных
	        \pause
            \item	\textit{Фильтры изменений}
        \end{itemize}
    \end{frame}
    
    \begin{frame}\frametitle{Проблемы (клиентская часть)}
        \begin{itemize}%[<+->]
            \item Особенности кросс-платформенной разработки в Qt \\
            (Linux / Mac OS X / Windows)
            \item Особенности Qt 5 под Ubuntu (Unity)
        \end{itemize}
    \end{frame}
    
    \begin{frame}\frametitle{Результаты (клиентская часть)}
        \begin{itemize}%[<+->]
            \item Разработано кроссплатформенное ядро клиента            	
            \begin{itemize}
                \item работа с сетью
                \item простой HTTP-сервер
                \item классы для выполнения основных действий
            \end{itemize}
            \item Консольный прототип клиента для тестирования функциональности
            \item Клиент с графическим интерфейсом под \\ OS X, Ubuntu, Windows
            \begin{itemize}
                \item общая кроссплатформенная часть
                \item платформозависимая часть --- вывод уведомлений
                \item поддерживает сохранение состояния (авторизация, подписка
                на уведомления) между запусками
            \end{itemize}
        \end{itemize}
    \end{frame}

    \begin{frame}\frametitle{Полученные знания}
        \begin{itemize}%[<+->]
            \item	Изучены основные принципы разработки клиент-серверных приложений
            \begin{itemize}
                \item механизм OAuth2 авторизации
                \item механизм построения и разработки веб-сервисов
            \end{itemize}            
            \item	Изучена связка GoogleAppEngine + webApp2 + Google Database API
            \item	Тонкие моменты кроссплатформенных приложений на Qt
        \end{itemize}
    \end{frame}

    \begin{frame}
        \begin{center}
            Спасибо за внимание!
        \end{center}
    \end{frame}

\end{document}
